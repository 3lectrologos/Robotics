\documentclass[a4paper, 11pt, english, greek]{article}

\usepackage{babel}
\usepackage{ucs}
\usepackage[utf8x]{inputenc}

\usepackage[T1]{fontenc}
%%\usepackage{lmodern}
\renewcommand{\ttdefault}{pcr}

\usepackage{subfig}
\usepackage[pdftex]{graphicx}
%%\usepackage{cite}
\usepackage{latexsym}

\title{Ρομποτική ΙΙ: Ευφυή Ρομποτικά Συστήματα \\ \vspace{12pt}
Εργασία 1: Κινηματικός έλεγχος ρομπότ με πλεονάζοντες βαθμούς ελευθερίας}
\author{Άλκης Γκότοβος}

\usepackage{hyperref}
\usepackage[all]{hypcap}

\begin{document}

\begin{titlepage}
	\maketitle
	\thispagestyle{empty}
\end{titlepage}

\section{Στόχος}
Στόχος της άσκησης είναι η δημιουργία περιβάλλοντος προσομοίωσης ενός ρομποτικού βραχίονα με 4 βαθμούς ελευθερίας.

Ο βραχίονας θα εκτελεί μία \emph{πρωτεύουσα εργασία}, η οποία συνίσταται στην παραμονή του τελικού στοιχείου
δράσης πάνω σε μία κατακόρυφη ευθεία.
Παράλληλα θα εκτελείται και μία \emph{δευτερεύουσα εργασία}, η οποία συνίσταται στην αποφυγή δύο κυκλικών εμποδίων,
τα οποία θα μπορούν να κινηθούν από το χρήστη με το πάτημα δύο κουμπιών.

\section{Γενικό σχήμα ελέγχου}
Για την προσομοίωση της κίνησης του ρομποτικού βραχίονα, θα χρησιμοποιηθεί το \emph{σχήμα ελέγχου τροχιάς επιλυμένης
ταχύτητας}.

Το γεωμετρικό μοντέλο του βραχίωνα με 4 παράλληλες στροφικές αρθρώσεις είναι απλό και δε θα αναφερθεί εδώ.

Το κομβικό σημείο στο έλεγχο επιλυμένης ταχύτητας είναι η λύση του αντιστρόφου κινηματικού προβλήματος, δηλαδή
η εύρεση των απαιτούμενων ταχυτήτων των αρθρώσεων, δεδομένης της επιθυμητής ταχύτητας του τελικού στοιχείου δράσης.
Στις επόμενες δύο ενότητες θα δείξουμε πώς επιλύουμε το πρόβλημα αυτό, ώστε να πετύχουμε την εκτέλεση των δύο
επιθυμητών εργασιών.

\section{Πρωτεύουσα εργασία}
Για την εκτέλεση της πρωτεύουσας εργασίας απαιτείται το τελικό στοιχείο δράσης να παραμένει επί μιας κατακόρυφης
ευθείας, ενώ ο προσανατολισμός δεν παίζει ρόλο.

Αυτό σημαίνει ότι αρκεί ένας βαθμός ελευθερίας για την εργασία αυτή.
Επιπλέον, επειδή η ευθεία είναι κατακόρυφη, ο κινηματικός περιορισμός του τελικού στοιχείου δράσης τίθεται μόνο
για τη συνιστώσα \textlatin{x}.
Συνεπώς, ο Ιακωβιανός πίνακας της πρωτεύουσας εργασίας θα αφορά μόνο την προαναφερθείσα συνιστώσα, δηλαδή θα είναι
ο πίνακας-γραμμή

\begin{center}
	$\mathbf{J_{1}} = [\frac{\partial p_{x}}{\partial q_{1}} \frac{\partial p_{x}}{\partial q_{2}}
	                   \frac{\partial p_{x}}{\partial q_{3}} \frac{\partial p_{x}}{\partial q_{4}}]$
\end{center}
Τότε, η λύση του αντιστρόφου κινηματικού προβλήματος, που ικανοποιεί τον περιορισμό της πρωτεύουσας εργασίας,
έχει τη μορφή

\begin{center}
	$\mathbf{\dot{q}} = \mathbf{J_{1}^{+}} \mathbf{\dot{p}} +
	         (\mathbf{I} - \mathbf{J_{1}^{+}} \mathbf{J_{1}}) \mathbf{\dot{q}_{0}}$
\end{center}
Ο πίνακας $\mathbf{J_{1}^{+}}$ είναι ο ψευδοαντίστροφος του $\mathbf{J_{1}}$.
To $\mathbf{\dot{q}_{0}}$ μπορεί να είναι οποιοδήποτε διάνυσμα, αφού ο δεύτερος όρος του παραπάνω αθροίσματος θα ανήκει
σε κάθε περίπτωση στο μηδενικό χώρο της πρωτεύουσας εργασίας.

\end{document}